% Created 2019-05-28 mar 10:29
% Intended LaTeX compiler: pdflatex
\documentclass[11pt]{article}
\usepackage[utf8]{inputenc}
\usepackage[T1]{fontenc}
\usepackage{graphicx}
\usepackage{grffile}
\usepackage{longtable}
\usepackage{wrapfig}
\usepackage{rotating}
\usepackage[normalem]{ulem}
\usepackage{amsmath}
\usepackage{textcomp}
\usepackage{amssymb}
\usepackage{capt-of}
\usepackage{hyperref}
\usepackage[margin=2cm]{geometry}
\author{javier}
\date{\today}
\title{}
\hypersetup{
 pdfauthor={javier},
 pdftitle={},
 pdfkeywords={},
 pdfsubject={},
 pdfcreator={Emacs 26.1 (Org mode 9.1.9)}, 
 pdflang={English}}
\begin{document}


\section{Radio de curvatura}
\label{sec:org64b49bf}

De Wikipedia, la enciclopedia libre

El \textbf{radio de curvatura} es una magnitud que mide la
curvatura de un objeto geométrico tal como una
línea curva, una superficie o más en general una
variedad diferenciable embebida en un
espacio euclídeo.

\subsection{Radio de curvatura de una curva}
\label{sec:orgb0eed8d}

El \textbf{radio de curvatura} de una línea curva o un objeto
aproximable mediante una curva es una magnitud geométrica que puede
definirse en cada punto de la misma y que coincide con el inverso del
valor absoluto de la curvatura en cada punto:

\begin{quote}
\(R_{c}(s):=\frac{1}{\chi(s)}\)
\end{quote}

Por otro lado la curvatura es una medida del cambio que sufre la
dirección del vector tangente a una curva cuando nos movemos a lo largo
de ésta. Para una curva parametrizada cualquiera la curvatura y el radio
de curvatura vienen dados por:\footnote{Spiegel, M. \& Abellanas, 1988, p. 121} 

\begin{quote}
\(\frac{1}{R_{c}(t)} = \chi(t) = \frac{\left| \dot{\mathbf{r}}(t) \times \ddot{\mathbf{r}}(t) \right|}{\left| \dot{\mathbf{r}}(t) \right|^{3}}\)
\end{quote}

Si en lugar de un parámetro cualquiera usamos el parámetro de
longitud de arco, la anterior ecuación se
simplifica mucho, por resultar un vector tangente constante, y puede
escribirse como:

\begin{quote}
\(\frac{1}{R_{c}(s)} = \chi(s) = \left| {{\overset{\sim}{\mathbf{r}}}^{''}(s)} \right|\)
\end{quote}

\subsubsection{Curvas planas}
\label{sec:org567b15d}

Para una curva plana cuya ecuación pueda escribirse en coordenadas
cartesianas
\$(x,y)$\backslash$,\$
como
\(x = x(t);y = y(t)\)
donde t es un parámetro arbitrario, la expresión para el radio de
curvatura se reduce a:

\(R_{c} = \frac{\left\lbrack {\left( \frac{dx}{dt} \right)^{2} + \left( \frac{dy}{dt} \right)^{2}} \right\rbrack^{\frac{3}{2}}}{\left| {\frac{dx}{dt}\frac{d^{2}y}{dt^{2}} - \frac{dy}{dt}\frac{d^{2}x}{dt^{2}}} \right|}\)

En caso que pueda escribirse
\$y = f(x)$\backslash$,\$
de tal modo que para cada punto de la curva exista un único valor de
\(x\)
entonces puede tomarse a
\(x\)
como el parámetro arbitrario, y el radio de curvatura se puede calcular
simplemente como:

\begin{quote}
\(R_{c} = \frac{\left\lbrack {1 + \left( \frac{df}{dx} \right)^{2}} \right\rbrack^{\frac{3}{2}}}{\left| \frac{d^{2}f}{dx^{2}} \right|}\)
\end{quote}

\subsubsection{Demostración}
\label{sec:org87e035a}

En primer lugar tenemos la ecuación paramétrica de la curva
\(\gamma(t):\mathbb{R}\rightarrow\mathbb{R}^{n}\)
de la que queremos deducir su radio de curvatura
\(\rho\)
Ahora debemos buscar la ecuación paramétrica de la circunferencia
(\(g:\mathbb{R}\rightarrow\mathbb{R}^{n}\)
que toma el mismo valor que
\(\gamma\)
y además satisfaga que
\(g^{\prime}(t) = \gamma^{\prime}(t)\)
y
\(g^{''}(t) = \gamma^{''}(t)\)
para cada
\(t\)
fijado. Claramente el radio no depende de la posición
(\(\gamma(t)\)
solo de la velocidad
(\(\gamma^{\prime}(t)\)
y la aceleración
(\(\gamma^{''}(t)\)
A partir de dos vectores
\(v\)
y
\(w\)
solo se pueden obtener tres escalares independientes, que son:
\(v \cdot v\)
\(w \cdot w\)
y
\(w \cdot v\)
. Por lo tanto el radio de curvatura dependerá únicamente de los
escalares
\(|\gamma^{\prime}(t)|^{2}\)
y
\(\gamma^{\prime}(t) \cdot \gamma^{''}(t)\)
.

La ecuación paramétrica general para una circunferencia en
\(\mathbb{R}^{n}\)
viene dada por

\(g(u) = A\cos\left( {h(u)} \right) + B\sin\left( {h(u)} \right) + C\)

donde
\(C \in \mathbb{R}^{n}\)
es el centro de la circunferencia (aunque es irrelevante, por
desaparecer al derivar),
\(A,B \in \mathbb{R}^{n}\)
son vectores perpendiculares le módulo
\(\rho\)
(es decir
\(A \cdot A = B \cdot B = \rho^{2} \land A \cdot B = 0\)
y
\(h:\mathbb{R}\rightarrow\mathbb{R}\)
es una función cualquiera doblemente diferenciable en
\(t\)

Derivando

\begin{quote}
\(\begin{matrix}
  {|g^{\prime}|^{2}} & = & {\rho^{2}(h^{\prime})^{2}} \\
  {g^{\prime} \cdot g^{''}} & = & {\rho^{2}h^{\prime}h^{''}} \\
  {|g^{''}|^{2}} & = & {\rho^{2}\left( {(h^{\prime})^{4} + (h^{''})^{2}} \right)} \\
  \end{matrix}\)
\end{quote}

si ahora igualamos a las derivadas correspondientes de
\(\gamma\)
obtenemos

\begin{quote}
\(\begin{matrix}
  {|\gamma^{\prime 2}(t)|} & = & {\rho^{2}h^{\prime 2}(t)} \\
  {\gamma^{\prime}(t) \cdot \gamma^{''}(t)} & = & {\rho^{2}h^{\prime}(t)h^{''}(t)} \\
  {|\gamma^{''2}(t)|} & = & {\rho^{2}(h^{\prime 4}(t) + h^{''2}(t))} \\
  \end{matrix}\)
\end{quote}

que se trata de un sistema en
\(\rho\)
\(h^{\prime}(t)\)
y
\(h^{''}(t)\)
que permite despejar
\(\rho\)
obteniendo finalmente que

\(\rho = \frac{|\gamma^{\prime}|^{3}}{\sqrt{|\gamma^{\prime}|^{2}\;|\gamma^{''}|^{2} - (\gamma^{\prime} \cdot \gamma^{''})^{2}}}\)
.

\subsection{Referencias}
\label{sec:orge976f51}

\begin{enumerate}
\item Spiegel, M. \& Abellanas, 1988, p. 121
\end{enumerate}

\subsubsection{Bibliografía}
\label{sec:org74ba532}

\begin{itemize}
\item Girbau, J.: "\emph{Geometria diferencial i relativitat}", Ed. \href{file:///wiki/Universidad\_Aut\%C3\%B3noma\_de\_Barcelona}{Universitat Autònoma de Barcelona}, 1993.
ISBN 84-7929-776-X.
\item Spiegel, M. \& Abellanas, L.: "\emph{Fórmulas y tablas de matemática aplicada}", Ed. McGraw-Hill, 1988.
ISBN 84-7615-197-7.
\end{itemize}

\subsubsection{Enlaces}
\label{sec:org76d22f8}
externoseditar]
:CUSTOM\(_{\text{ID}}\): enlaces-externoseditar

\begin{itemize}
\item Weisstein, Eric W.
\href{http://mathworld.wolfram.com/PrincipalCurvatures.html}{«Principal Curvatures»}. En Weisstein, Eric W. \emph{MathWorld}
(en inglés). Wolfram Research. 
\item Weisstein, Eric W.
\href{http://mathworld.wolfram.com/PrincipalRadiusofCurvature.html}{«Principal Radius of Curvature»}. En Weisstein, Eric W.
\emph{MathWorld} (en inglés).
Wolfram Research. 
\item \href{http://proyectodescartes.org/uudd/materiales\_didacticos/curvatura-JS/index.htm}{Introducción a la curvatura de curvas planas.}
\end{itemize}

\subsubsection{Licencia}
\label{sec:orgad53468}

El texto está disponible bajo la
\href{https://es.wikipedia.org/wiki/Wikipedia:Texto\_de\_la\_Licencia\_Creative\_Commons\_Atribuci\%C3\%B3n-CompartirIgual\_3.0\_Unported}{Licencia Creative Commons Atribución Compartir Igual 3.0};
pueden aplicarse cláusulas adicionales.
\end{document}
