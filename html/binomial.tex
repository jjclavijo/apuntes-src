% Created 2019-05-28 mar 10:50
% Intended LaTeX compiler: pdflatex
\documentclass[11pt]{article}
\usepackage[utf8]{inputenc}
\usepackage[T1]{fontenc}
\usepackage{graphicx}
\usepackage{grffile}
\usepackage{longtable}
\usepackage{wrapfig}
\usepackage{rotating}
\usepackage[normalem]{ulem}
\usepackage{amsmath}
\usepackage{textcomp}
\usepackage{amssymb}
\usepackage{capt-of}
\usepackage{hyperref}
\usepackage[margin=2cm]{geometry}
\author{javier}
\date{\today}
\title{}
\hypersetup{
 pdfauthor={javier},
 pdftitle={},
 pdfkeywords={},
 pdfsubject={},
 pdfcreator={Emacs 26.1 (Org mode 9.1.9)}, 
 pdflang={English}}
\begin{document}


\section{Binomial series}
\label{sec:orgfb57b45}

The \textbf{binomial series} is the Maclaurin series for the function
\(f\)
given by
\(f(x) = (1 + x)^{\alpha}\)
where
\(\alpha \in \mathbb{C}\)
is an arbitrary complex number. Explicitly,

$$\begin{matrix}
{(1 + x)^{\alpha}} & {= \sum\limits_{k = 0}^{\infty}\;{(\frac{\alpha}{k})}\; x^{k}\qquad\qquad\qquad(1)} \\
 & {= 1 + \alpha x + \frac{\alpha(\alpha - 1)}{2!}x^{2} + \cdots,} \\
\end{matrix}$$

and the binomial series is the power series on
the right hand side of (1), expressed in terms of the
(generalized) binomial coefficients

$${(\frac{\alpha}{k})}:=\frac{\alpha(\alpha - 1)(\alpha - 2)\cdots(\alpha - k + 1)}{k!}$$

\subsection{Special cases}
\label{sec:orgecbf178}

If \(\alpha\) is a nonnegative integer \(n\), then the \((n + 2)^{th}\) term and all
later terms in the series are 0, since each contains a factor
\(n-n\); thus in this case the series is finite and gives the
algebraic binomial formula.

The following variant holds for arbitrary complex \(\beta\), but is especially
useful for handling negative integer exponents in (1):

$$\frac{1}{(1 - z)^{\beta + 1}} = \sum\limits_{k = 0}^{\infty}{(\frac{k + \beta}{k})}z^{k}.$$

To prove it, substitute \(x=-z\) in (1) and apply a binomial
coefficient identity, which is,

$${(\frac{- \beta - 1}{k})} = ( - 1)^{k}{(\frac{k + \beta}{k})}.$$

\subsection{Summation of the binomial series}
\label{sec:org83a9472}

The usual argument to compute the sum of the binomial series goes as
follows. Differentiating term-wise the binomial series within the
convergence disk \(|x| < 1\) and using formula (1), one has that the sum
of the series is an analytic function
solving the ordinary differential equation \((1+x)u'(x) = \alpha u(x)\)
with initial data \(u(0) = 1\). The unique solution of this
problem is the function \(u(x) = (1+x)^{\alpha}\), which is therefore
the sum of the binomial series, at least for \(|x|<1\). The equality
extends to \(|x|=1\) whenever the series converges, as a consequence of
Abel's theorem and by continuity of \((1+x)^{\alpha}\).
\end{document}
